%%%%%%%%%%%%%%%%%%%%%%%%%%%%%%%%%%%%%
\part{Preliminaries}
%%%% Disclaimer
\normalfont
\begin{frame}
	\begin{center}
  		\begin{block}{Disclaimer} 
The author assumes you are not familiar with \ttfamily R \normalfont and need not have programmed before. \\
		\end{block}
	\end{center} 
\end{frame}

\begin{frame}
	\begin{center}
  		\begin{block}{Why R?} 
			\begin{itemize}
				\item Absolutely free!
				\item Used in industry and academia.
				\item Has a great community:
					\begin{itemize}
						\item StackOverflow
						\item Blogs
						\item Meetup groups
						\item MOOCs
						\item many, many others
					\end{itemize}
				\item Has over 7500 packages available for use (for free!).
				\item Transparent code (e.g. easier to check for bugs).
			\end{itemize}
		\end{block}
	\end{center} 
\end{frame}

\begin{frame}
	\begin{center}
  		\begin{block}{Overview of R per John Chambers [1]:} 
			"To understand computations in R, two slogans are helpful:
			\begin{itemize}
			        \item Everything that exists is an object.
			        \item Everything that happens is a function call."
			\end{itemize}
		\end{block}
	\end{center} 

% R is flexible: 
% 	\begin{itemize}
% 		\item Objects can be variables, data sets, etc. and can be self-created, downloaded off the web (or elsewhere) and loaded from package(s).
% 		\item Functions (which do something to objects) can also be self-created, downloaded off the web (or elsewhere) and loaded from package(s).
% 		\item A package (1 of 7500+) is a collection of data sets and/or functions unified with a common theme.
% 	\end{itemize}
\end{frame}

\begin{frame}
	\begin{center}
  		\begin{block}{Goal of the tutorial} 
Visualizations:
	\begin{itemize}
		\item what plot to use to examine a variable or variables?
		\item how easy to use is R for visualizations?
		\item how flexible is R for visualizations?
	\end{itemize}		
		\end{block}
	\end{center} 
\end{frame}

%%%%%%%%%%%%%%%%%%%%%%%%%%%%%%%%%%%%%%

%%%%%%%%%%%%%%%%%%%%%%%%%%%%%%%%%%%%%%


%%%%%%%%%%%%%%%%%%%%%%%%%%%%%%%%%%%%%%

%% Subsection: Importing Data Sets into R

%\subsection[Importing Data]{Importing Data Sets into R}

%%%% Subsection: Importing Data from the Internet

%\subsubsection{Importing Data from the Internet}

%%%%%%%%%%%%%%%%%%%%%%%%%%%%%%%%%%%%%%
%% --------------------------------------------------- Slide --
%\begin{frame}[fragile]
%  \frametitle{Data from the Internet}

% When downloading data from the internet, use \ttfamily read.table(). \normalfont  In the arguments of the function:
%   \begin{itemize}
%   \item \ttfamily header: \normalfont if TRUE, tells R to include variables names when importing
%   \item \ttfamily sep: \normalfont tells R how the entires in the data set are separated
%     \begin{itemize}
%       \item \ttfamily sep=",": \normalfont when entries are separated by COMMAS
%       \item \ttfamily sep="$\backslash t$": \normalfont when entries are separated by TAB
%       \item \ttfamily sep=" ": \normalfont when entries are separated by SPACE
%     \end{itemize}
%    \end{itemize}
%    	\begin{lstlisting}
%data<-read.table("http://www.stat.ucla.edu
%/~vlew/stat130a/datasets/twins.csv", 
%header=TRUE, sep=",")
%	\end{lstlisting}
%\normalfont
%\normalsize
%\end{frame}

%%%%%%%%%%%%%%%%%%%%%%%%%%%%%%%%%%%%%%

%% Data from your Computer

%\subsubsection{Importing Data from Your Computer}

%%%%%%%%%%%%%%%%%%%%%%%%%%%%%%%%%%%%%%
%\begin{frame}[fragile]
% \frametitle{Importing Data from Your Computer}
%     \begin{enumerate}
%  	\item Check what folder R is working with now: \\
%		\begin{lstlisting}
%getwd()
%		\end{lstlisting}

%  	\item Tell R in what folder the data set is stored (if different from (1)).  Suppose your data set is on your desktop: \\
%	\begin{lstlisting}
%setwd("~/Desktop")
%	\end{lstlisting}

%	\item Now use the \ttfamily read.table() \normalfont command to read in the data, substituting the name of the file for the website.
%     \end{enumerate}
%\end{frame}

%%%%%%%%%%%%%%%%%%%%%%%%%%%%%%%%%%%%%%

%% Data Available in R
%\subsubsection{Using Data Available in R}

%%%%%%%%%%%%%%%%%%%%%%%%%%%%%%%%%%%%%%
%\begin{frame}[fragile]
%  \frametitle{Using Data Available in R}
%\begin{enumerate}
%\item To use a data set available in one of the R packages, install that package (if needed).

%\item Load the package into R, using the \ttfamily library() \normalfont function. \\
%	\begin{lstlisting}
%library(alr3)
%	\end{lstlisting}

%\item Extract the data set you want from that package, using the \ttfamily data() \normalfont function.  In our case, the data set is called \ttfamily UN2. \\
%	\begin{lstlisting}
%data(UN2)
%	\end{lstlisting}

%\end{enumerate}
%\end{frame}

